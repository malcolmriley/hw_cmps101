\begin{proof}
It must be demonstrated that for any real constants $a$ and $b$ that if $b > 0$, then $(n+a)^b = \Theta(n^b)$.
The following properties will be used:
\begin{enumerate}
	\itemlabel{prop:theta}{Property of $\Theta(f(n))$ for Polynomials}
		For a polynomial $f(n)$ whose leading term is $a \cdot n^b$ for some real numbers $a$ and $b$, $f(n) = \Theta(n^b)$.
	\itemlabel{thm:binom}{The Binomial Theorem}
		\begin{equation*}
			(a+b)^n = \sum_{k=0}^n \binom{n}{k} a^{n-k} b^k
		\end{equation*}
\end{enumerate}
\begin{itemize}
	\item By theorem \ref{thm:binom}, the expression $(n + a)^b$ evaluates to a polynomial whose leading term is $ \binom{n}{0} n^{b-0} a^0 $
	\item The polynomial obtained previously may be evaluated to a polynomial of leading term $ n^b $. Let this polynomial be represented by $f(n^b)$.
	\item By property \ref{prop:theta}, $ f(n^b) = \Theta(n^b) $.
	\item Thus, the desired relation is established.
\end{itemize}
\end{proof}