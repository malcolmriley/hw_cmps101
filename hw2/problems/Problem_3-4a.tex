It cannot be true that $ f(n) = O(g(n)) \implies g(n) = O(f(n)) $ since by the definition of $ O(f(n)) $ it is only known that there exists some $ c $ such that $ 0 \leq f(n) \leq c \cdot g(n) $. It is not necessarily known from $ f(n) = O(g(n)) $ that there is some $ d $ such that $ 0 \leq g(n) \leq d \cdot f(n) $ (as $ g(n) = O(f(n)) $ would require), hence the implication does not follow. For a specific counterexample: $ n^2 = O(n) $, but $ n \neq O(n^2) $.