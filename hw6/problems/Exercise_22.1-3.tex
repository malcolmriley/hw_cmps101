For an adjacency-list representation of a graph $ G $, computing $ G^T $ would involve traversing every list contained by $ G $ and adding the discovered vertices as entries in a new graph. If an adjacency list of $ G $ is backed by a number of lists $ L_v $ such that an entry $ e $ of $ L_v $ means that $ v $ is adjacent to $ e $, then to calculate $ G^T $ it is needed to each list $ L_v $ for values $ e $, and insert $ v $ into $ L_e $ of the new graph. This takes a number of operations equal to the total number of edges of the original graph, thus $ O ( \mid E \mid $).
For an adjacency-matrix representation of a graph $ G $, computing $ G^T $ is of the same computational complexity as computing the transpose of the backing matrix. It is necessary to swap every entry $ (i,j) $ with $ (j, i) $ of matrix $ M $, except for the diagonal ( where that $ i = j $ ). If $ M $ is an $ n \times n $ matrix, then this will take $ n \cdot (n - 1) $ operations; thus $ O (n^2 - n) $.