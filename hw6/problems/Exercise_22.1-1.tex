For an adjacency-list representation of a directed graph, the time taken to compute the out-degree property of a vertex $ v $ is simply the time taken to determine the length of the list corresponding to $ v $. If the length of the list is stored as a property of the list, determining the length of the list will be a constant-time operation and thus the computational time will be $ O ( \mid V \mid ) $. If the length of the list is not stored as a property of the list, then it will be necessary to count the elements of the list; for a maximally connected graph, the length of each adjacency list will be $ \mid V \mid - 1 $, as each vertex $ v $ is adjacent to all other vertices except for itself; thus the time taken to calculate the out-degree of every vertex will be at most $ O ( \mid V \mid^2 + 1) $. Calculating the in-degrees of the entire matrix is a similar prospect: it is necessary to count the number of occurrences of each value $ v $ in each adjacency list in the graph object. This entails a complete traversal of each adjacency list - as given previously, this will take $ O ( \mid V \mid^2 + 1) $ operations.