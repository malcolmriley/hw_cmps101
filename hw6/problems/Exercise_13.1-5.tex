By property 5, any two simple paths in a red-black tree from a given node to any descendent leaf will have the same number of black nodes. Let $ n $ represent this quantity of black nodes from the parent node $ p $ to any indicated descendent $ d $. The shortest possible path between $ p $ and $ d $ would be one consisting entirely of black nodes; the length of this path is thus $ n $. As a corollary to property $ 5 $, there might exist at most one node between every two black nodes in any given path in a red-black tree. Thus, the longest possible path between $ p $ and $ d $ would be one where every black node is preceded (or succeeded) by a red node; the length of this path will be the number of black nodes $ n $ plus the number of red nodes $ r $. In this longest-possible path, since there is one red node for every black node, $ n = r $, and thus the length of this path is $ (n + r) = (n + n) = (2n) $. Thus, for any two simple paths in a red black tree, the longest possible path is at most twice the length of the shortest possible path.