By the definition of simple uniform hashing, for a hash function $ h $ using a finite number $ n $ distinct keys into an array with finite length $ m $, each index $ m $ is equally likely to be returned by $ h $ for a given value of $ n $. In other words, for any value $ n_i $, the likelihood that $ h(n_i) = m_i $ is $ \frac{1}{m} $. A collision for two keys $ n_1 $ and $ n_2 $ means that $ h(n_1) = h(n_2) $. If $ n < m $ and the distribution of hashes is truly uniform there should be no collisions, as there are more values of $ h(n) $ than values of $ n $. However, if $ n > m $ then there will be $ n - m $ collisions.