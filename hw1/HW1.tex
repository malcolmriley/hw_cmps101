
\documentclass[11pt]{article}
\usepackage{listings}
\usepackage{color}
\usepackage{amsthm}
\usepackage{amsmath}
\usepackage{csvsimple}
\usepackage{graphicx}
\usepackage{geometry} % see geometry.pdf on how to lay out the page. There's lots.
\geometry{letterpaper} % or letter or a5paper or ... etc

\definecolor{light_grey}{rgb}{0.95,0.95,0.95}

\lstset{
	title={Pseudocode:},
	backgroundcolor=\color{light_grey},
	basicstyle=\footnotesize\ttfamily,
}

\title{\textbf{Homework 1}}
\author{Malcolm Riley}
\date{April 2018}

%%% BEGIN DOCUMENT
\begin{document}

\maketitle
\pagenumbering{arabic}

\section*{Exercise 1.2-2}
For values of $n$ less than $6$, insertion sort beats merge sort.

\section*{Exercise 1.2-3}
$15$ is the smallest value of $n$ such that the algorithm whose runtime is $100n^2$ is better.

\section*{Problem 1-1}
\csvautotabular{table1-1_1.csv} \\ \\
\csvautotabular{table1-1_2.csv} \\
Calculations assume a month of 30 days, and a year of 365 days.

\section*{Exercise 2.1-1}
The action of the algorithm is as follows:
\begin{enumerate}
\item { $ \{31,41,59,26,41,58\} $ }
\item { $ \{26,31,41,59,41,58\} $ }
\item { $ \{26,31,41,59,41,58\} $ }
\item { $ \{26,31,41,59,41,58\} $ }
\item { $ \{26,31,41,41,59,58\} $ }
\item { $ \{26,31,41,41,58,59\} $ }
\end{enumerate}

\section*{Exercise 2.1-3}
\lstinputlisting{pseudo2.1-3.txt}

\textbf{Loop Invariant:} For each value $i$, there is no index $x < i$ such that $A[i] = v$.
\textbf{Initialization:} When initializing, $i = 0$. There is no value of $x$ less than $0$ such that $A[x] = v$.
\textbf{Maintenance:} The loop continues if, for the current value of $i$, $A[i]$ does not equal $v$. During the next iteration, the value of $i$ will be $(i+1)$. Therefore, for each incrementing value of $i$, there is not a value $x$ less than $i$ such that $A[x] = v$.
\textbf{Termination:} The algorithm will terminate and return a value under two conditions:

\begin{itemize}
	\item If all valid values of $i$ have been exhausted; that is to say, a value $x$ such that $A[x] = v$ is not found in the range between $0$ and the length of $A$, the loop will exhaust and the algorithm will return a value of $NIL$.
	\item If a value of $i$ is found such that $A[i] = v$, the loop will return $i$, the first index of $A$ such that $A[i] = v$.
\end{itemize}

\section*{Exercise 2.2-1}
$\Theta(n^3)$, since the leading term of the polynomial $n^3/1000 - 100n^2 - 100n + 3$ is $n^3$.

\section*{Exercise 2.2-2}
\lstinputlisting{pseudo2.2-2.txt}

\textbf{Loop Invariant:} The first $n$ items of $A$ form a sorted subarray $B$ such that every element in $B$ is less than every element in $A$.

The algorithm only needs to run for the first $n-1$ elements, because the $n^{th}$ element always has a greater value than all of the first $n-1$; therefore, by the time the algorithm gets to the $n^{th}$ element, it is known that it is the greatest-value element in the array.

The best and worst case for selection sort is $\Theta(n^2)$ - in all cases, comparisons and swaps over the entire array need to be made.

\section*{Exercise 2.2-3}
For an array of size $n$, in the average case it will need to search half the array; therefore $\Theta(n/2)$. In the worst case, it will have to search the entire array, therefore $\Theta(n)$.

\section*{Exercise 2.3-1}
The action of the algorithm is as follows:
\begin{enumerate}
	\item { $ \{3,41,52,26,38,57,9,49\} $ }
	\item { $ \{3,41,52,26\} \{38,57,9,49\} $ }
	\item { $ \{3,41\} \{52,26\} \{38,57\} \{9,49\} $ }
	\item { $ \{3\} \{41\} \{52\} \{26\} \{38\} \{57\} \{9\} \{49\} $ }
	\item { $ \{3,41\} \{26,52\} \{38,57\} \{9,49\} $ }
	\item { $ \{3,26,41,52\} \{9,38,49,57\} $ }
	\item { $ \{3,9,26,38,41,49,52,57\} $ }
\end{enumerate}


\section*{Exercise 2.3-3}
It must be shown that $T(n) = n \lg(n)$ for all $n$.
\begin{proof}
	It is known that for all $n$, $n = 2^k$ for some integer $k$, since $n$ is a power of $2$.
	\begin{itemize}
		\item{ \textbf{Base Case:} For $n = 2$: $T(n) = T(2) = 2 = 2 \lg(2)$, thus the equality holds. }
		\item{ \textbf{Assume:} $T(n) = n \lg(n)$ }
		\item{ \textbf{Let $n = 2^{k+1}$.} }
		\begin{align*}
		T(2^{k+1}) &= 2T(\frac{2^{k+1}}{2}) + 2^{k+1} \\
		" &= 2T(2^k) + 2^{k+1} \\
		" &= 2(2^k\lg(2^k)) + 2^{k+1} \\
		" &= k 2^{k+1} + 2^{k+1} \\
		" &= (k+1)2^{k+1} \\
		" &= 2^{k+1}\lg(2^{k+1}) \\
		\end{align*}
		\item{ Letting $m = k+1$ demonstrates the required equality: $ T(m) = m \lg(m) $}
	\end{itemize}
\end{proof}

\section*{Exercise 2.3-5}
\lstinputlisting{pseudo2.3-5.txt}

\section*{Exercise 2.3-6}
Insertion sort, regardless of this optimization, still iterates forward over the unsorted segment of the array, meaning that the runtime is at least $n$, the length of the array. For each element $n$, some sorting procedure needs to take place; performing a binary search on the sorted portion of the array means that for every element $n$, a runtime of at worst $\lg(n)$ takes place. This brings the overall asymptotic runtime of the improved Insertion Sort procedure to $\Theta(n\lg(n))$.

\end{document}